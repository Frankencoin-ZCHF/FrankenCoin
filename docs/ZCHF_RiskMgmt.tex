\documentclass[english,11pt]{article}
\usepackage{url}
\usepackage{threeparttable}
\usepackage{geometry}
\usepackage{lscape}
%\usepackage{mdwlis}

%%%%%%%%%%%%%%%% Reasonable Margins
\setlength{\textwidth}{6.5in} \setlength{\textheight}{9in} % 6.2 and 8.5 normal...
\setlength{\topmargin}{-1.1in} \setlength{\oddsidemargin}{0in}
\setlength{\parskip}{2mm}
\usepackage{paralist}
\usepackage{amsmath}
\usepackage{amssymb,epsfig,amsthm,amsfonts,bm}
\usepackage{graphicx}
%\usepackage[T1]{fontenc}
%\usepackage[ansinew]{inputenc}
%\usepackage{a4}
\usepackage{babel}
\usepackage{array}
%\usepackage{natbib}
\usepackage{cite}
%\usepackage{biblatex}
\usepackage{xr}
\usepackage{setspace}
\usepackage[small,bf]{caption}%[1995/04/05]
\usepackage[dvipsnames]{xcolor}
\usepackage{color}
\usepackage{dcolumn}
\usepackage{lscape}

\newcolumntype{.}{D{.}{.}{1}}
%\setlength{\abovecaptionskip}{-0.01cm}
%\setlength{\belowcaptionskip}{0.05cm}
%\usepackage[ruled,vlined]{algorithm2e}
%\usepackage{algorithm, algpseudocode}%loads algorithmicx
%\usepackage{algorithmicx}
\everydisplay{
\abovedisplayskip=.47\baselineskip  plus.2ex minus.2ex
\abovedisplayshortskip=-0.13\baselineskip plus.2ex minus.2ex
\belowdisplayskip=.37\baselineskip plus.2ex minus.2ex
\belowdisplayshortskip=.37\baselineskip plus.2ex minus.2ex
}

% nothing command
\newcommand{\nothing}[1]{}
\usepackage{arydshln}
\usepackage{booktabs}
\usepackage{array,hhline}
\usepackage{rotating}
\usepackage{tabularx}

\newcommand{\blue}[1]{{\color{RoyalBlue}#1}}

\usepackage{soul}
\setlength{\parskip}{1.5ex plus 0.5ex minus 0.5ex}

%\DeclareMathOperator{\Cov}{cov}
\newcommand{\E}{\mathbb{E}}    % E, expectation
\newcommand{\Var}{\mathsf{Var}}% Var, variance
\newcommand{\Cov}{\mathsf{Cov}}% Cov, covariance
\newcommand{\EWMA}{\mathsf{ewma}}% Cov, covariance
\usepackage{paralist}

% font
\usepackage[T1]{fontenc}
%\usepackage{lxfonts}


\definecolor{DarkBlue}{rgb}{0,0.0,0.5}
\definecolor{DarkGreen}{rgb}{0.01,0.4,0.01}

% THEOREMS
\newtheorem{theorem}{Theorem}[section]
\renewcommand\qedsymbol{$\blacksquare$}
\newtheorem{cor}{Corollary}
\newtheorem{lemma}{Lemma}
\newtheorem{proposition}{Proposition}
\newtheorem{defn}{Definition}
\newtheorem{rem}{Remark}
\newtheorem{assumption}{Assumption}
\newtheorem{hypo}{Hypothesis}

% <thanks symbol>
\makeatletter
\renewcommand*{\@fnsymbol}[1]{\ensuremath{\ifcase#1\or *\or **\or ***\or
   \mathsection\or \mathparagraph\or \|\or **\or \dagger\dagger
   \or \ddagger\ddagger \else\@ctrerr\fi}}
\makeatother
% </thanks symbol>

% short comment
\def\query#1{\marginpar{\begin{flushleft}\textcolor{red}{\footnotesize#1}\end{flushleft}}}%
% longer comment
\def\cmnt#1{ \textcolor{red}{#1} }
%\onehalfspacing
% rephrasing
\def\rephr#1{\textcolor{blue}{#1}}

%\onehalfspacing

\begin{document}
\title{An Automated Market Maker for On-Chain Perpetual Swaps\thanks{Frankencoin v1.0,
Document v0.1}}
\bigskip
\bigskip



{
\date{ \today  }


\author{
Basile Maire\thanks{\texttt{basile.maire@desma8.io}}\\}
}

\bigskip
\bigskip

\bigskip
\date{\today }




\newpage
\newpage
%\nothing{%blind
%>>>>>>>>>>>>>>>>>>>>\maketitle
\thispagestyle{empty}
%}%blind

%%%%%%%%%%%%%%%%%%%%%%%%%%%%%%%%%%%%%%%%%%%%%%%%%%%%%%%%%%%%%%%%%%%%%%%%%%%%%%%%%%%%%%%%%%%%%

\nothing{
	\newpage
	\thispagestyle{empty}
	
	\vspace{2.5cm}
	
	\thispagestyle{empty}
	
	\vspace{2.5cm}
	\smallskip
	\centerline{\bf {\Large Frankencoin - Risk Considerations}}
	
	\vspace{.5cm}
	\begin{abstract}
	\nopagebreak 
	\end{abstract}
	\bigskip
	\bigskip
	
	%\vspace{1cm}
	%\noindent{\it JEL Classification Codes: D40, G21, and L11}
	
	%\vspace{1cm}
	\bigskip
	%{{\noindent \it{Key Words}: 
	%}  }
	
	
	\bigskip

	\newpage
%>>>\tableofcontents

	\newpage
}
\setcounter{page}{1}

% !TEX root = ./perpetual.tex
%%%%%%%%%%%%%%%%%%%%%%%%%%%%%%%%%%%%%%%%%%%%%%%%%%%%%%%%%%%%%%%
\section{Fee Calibration}
We approach the calibration of minting fees similar to credit pricing. We calibrate the fees that we charge to the minters 
so that the expected losses to the stakers are paid for by fees.
The Liquid Collateral Contract, as outlined in the Frankencoin Manifesto, has the following features.
\begin{itemize}
\item The participants are minters, challengers, auction participants, and ZCHF stakers.
\item The minter deposits collateral and thereby mints ZCHF. The ZCHF are overcollateralized at the time
of minting, that is, the value of collateral deposited exceeds the value of the minted ZCHF.
\item Challengers can initiate an auction process for a given position at any time. Afterwards, auction participant
bid for the collateral.
\begin{enumerate}
\item If, according to the auction, the value of the collateral falls below a specified threshold, the position is liquidated
and the minter loses their collateral. \emph{E.g., the collateral deposited is 1500 LUSD, 1000 ZCHF were minted. Now, the position
is challenged and the best bid for 1500 LUSD closes at 1095 ZCHF. Let's assume that the threshold is 10\%. Now, because 
1095 ZCHF < 1000 (1+10\%) ZCHF, the position is closed out.}
\item If, according to the auction, the collateral value is above the threshold, the position remains in the
minter's ownership
\item If, according to the auction, the collateral value falls below the minted amount of ZCHF, ZCHF in the staking pool have to be burnt, i.e.,
a loss to ZCHF stakers.
\end{enumerate}
\end{itemize}

The system outlined above has the following parameters. Let $h$ be the threshold that defines whether the position
can be liquidated or not. That is, if, according to the auction, the value of the collateral is below 
$Z(1+h)$, where $Z$ is the amount of ZCHF minted for a given position, the position is liquidated.
Let $c$ be the challenger reward (e.g., 2\% of the ZCHF position). Finally, $\tau$ is the duration of the auction
(e.g., 3 days).
\begin{align}
h &: \text{$h>0$, liquidation threshold, liquidate if highest bid is below Z(1+h)}\\
\tau &: \text{duration of liquidation process}\\
c &: \text{$0<c<h$, challenger reward}
\end{align}

By $\tilde{r}$ we denote the random variable that corresponds to the log-return of the collateral using CHF as the quote-currency (or numeraire).
Now, we can express the loss to the stakers as the following random variable:
\begin{align}
\tilde{L} = -\min\left[(1 + h) e^{\tilde{r}} - (1+c), 0\right],
\end{align}
per unit of ZCHF minted (e.g., if the position consists of $Z$ ZCHF, the loss to the stakers is $Z\tilde{L}$).
To see this, first note that the starting value of the position is at least $(1 + h)$ per unit of $Z$. Since $\tilde{L}$ is per ZCHF, we set $Z=1$.
The value of the position
at the end of the auction is $Z(1 + h) e^{\tilde{r}}$. Stakers need to burn ZCHF for the amount that the end-of-auction value falls short
of the minted amount plus the challenger reward, hence we subtract $(1+c)Z$. We need the $\min[\cdot]$-function because
only losses are charged to the stakers, profits in the liquidation process are not distributed to stakers. Finally, the negative sign is 
convention to have a positive number for the loss.

If we assume that the auction process will result in prices similar to what we would observe on liquid exchanges,
then a remaining risk is that the price falls into a loss region during the auction period.
With a known density function for the return distribution over the period $\tau$, $f_{\tau}(x)$, we can calculate the expected loss as follows
\begin{align}
\mathbb{E}_{\tau}\left[\tilde{L} \right] &= - \int_{-\infty}^k \left((1 + h) e^{x} - (1+c) \right) f_{\tau}(x) dx\label{eq:E}\\
k &= \log \frac{1+c}{1+h},
\end{align}
where the subscript $\tau$ emphasizes that the distribution depends on the time-horizon of the auction.

\begin{table}[]
\caption{\textbf{Minting Fees.}
    We calibrate minting fees for a collateral with 5\% daily volatility,
    assuming a t-distribution with 4 degrees of freedom (left), or a normal
    distribution (right). The volatility is scaled by $\sqrt{\tau}$, and the
    mean is assumed to be zero. Values are in percentages.} % title of Table
	\begin{center}
	\begin{tabular}{lll}
	\hline
	\textbf{$\mathbf{\tau}$} & \textbf{t-distr} & \textbf{normal} \\
	\hline
	\textbf{1}                   & 0.38             & 0.08            \\
	\textbf{2}                   & 0.90             & 0.38            \\
	\textbf{3}                   & 1.38             & 0.72   \\        
	\hline
	\end{tabular}
	\label{tab:calibration}
	\end{center}
\end{table}

Table~\ref{tab:calibration} shows results
from calibrating Eq.~\eqref{eq:E}. BTC-USD has a daily volatility of about 5\%. 
The table compares the result for Eq.~\eqref{eq:E} when assuming
either a t-distribution or a normal distribution for the return.

\end{document}